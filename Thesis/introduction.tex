\documentclass[main.tex]{subfiles}

\begin{document}

\section{Introduction}
Barter has historically been infeasible on the large scale, and prior to money various methods of debt tallying were used \cite{Gra1}. One difficulty is that in barter both parties in an exchange must have a coincidence of wants simultaneously; money or debt allow a service to be offered before its equal is received. Another difficulty with barter is that exchanges involving more than two parties will be difficult to organize, whereas money allows someone to sell to one person and buy from another.

Money works well, yet not seamlessly. There are cases where people would rather exchange than use money. For example, by using Intervac \footnote{\url{www.intervac-homeexchange.com/}} \cite{Abr1} home-owners can exchange homes for the holidays, avoiding the hassles of renting their own home out for money. In addition, one could exchange for a home of equal market value; yet with money a home-owner is unlikely to get parity: there are taxes and middle-men involved at each step of the way, and the othre home-owner may also be seeking to rent at a profit.

Exchanging used goods may be preferable to selling for profit and using those profits to buy other used goods. Swap It \footnote{\url{https://www.swapit.la/}} \cite{Abr1} and SwapAce \footnote{\url{http://www.swapace.com}} \cite{Cab0}, for example, allow buying, selling, or swapping. BookMooch \footnote{\url{http://bookmooch.com/}} allows users to exchange books via points, essentiall a currency. Some people need different sized shoes for each foot and under normal circumstances need to buy two pairs, or an amputee would have to buy two shoes uselessly. Fortunately there is a National Odd Shoe Exchange \footnote{\url{http://oddshoe.org}} \cite{Abr1} to help all these individuals save money. Data can also be  Potentially, even groceries could be bartered locally: you have some extra bread and could offer some to a neighbor, if they only knew.

Tradebank \footnote{\url{http://tradebank.com}} provides barter matching services and has their own currency; however, they use humans to assist with matching rather than automation. Barter Network \footnote{\url{http://www.barternetworkinc.com}} provides a similar service. These services both show the value of matching users and helping them to find a coincidence of wants, even if currency is still used. There is significant corporate barter for various purposes: tax reduction, cooperation, handling surplus goods, and even avoiding economic transactions (e.g. Iran bartering oil for goods and assets) \cite{art:str}. Data can also be exchanged \cite{art:blo}.

Goods that cannot be sold for moral reasons \footnote{What Roth calls \textit{repugnant} transactions.}\cite{Rot3} are also exchanged using optimization algorithms. The best current example of this is kidney exchange programs in the US and UK, as the sale of kidneys is prohibited in every country but Iran.

There are already good use-cases when the matching and user-interface has to be custom-tailored to the item or service being exchanged. When possible, direct exchange is more efficient. eBay has 100 million items for sale each day \cite{ebay}. How many of them could be satisfied with an exchange?

The author and collaborators with the Global Brain Instiute \footnote{\url{https://sites.google.com/site/gbialternative1/}} have called such a general barter exchange network an Offer Network \cite{Goe1} \cite{Goe2} \cite{Hey1} \cite{Hey2}. The socioeconomic and political implications of a large scale global offer network are discussed in depth. Money, to the extent its still useful, will be grounded transparently in what people are willing to exchange with each other. People in developing countries will more easily enter the global market \footnote{This is already happening to some extent with the internet and cryptocoins.}. In Open Production Netorks \cite{Goe1} production lines can be made transparent and searchable, making anyone skimming off eploitative levels of profit readily apparent. As the marginal cost of daily and infrastructural needs trends towards zero, human exchanges will become dominated by I-You exchanges rather than monetary I-it ones\footnote{I-You interactions are personalized, whereas in I-it interactions one treats the other as an object.}. Internet of Things and 3D printer technology will make personalized designs more important than the materials, which will then be an attention service and more suited to exchange \cite{Hey1}. In fact, AI agents in internet of things technology may want to use an offer network of their own, or, a framework such as the Web of Needs \cite{Kle1} where agents upload there needs. The Web of Needs is design to allow external applications to perform services on agents needs around the globe.

While some instantiation of offer networks can be foreseen in the future, and can be predicted to have numerous subtle and profoundly beneficial impacts, the feasibility of satisfying users via exchange alone needs to be verified. This thesis sets out to test whether optimization algorithms can be used to overcome the second difficulty barter exchange faces: allowing and arranging k-way exchanges ($k>2$). If enough users can be satisfactorily matched without waiting too long, then offer networks will be feasible without more advanced artificial intelligence methods.

This thesis works with a minimal viable prototype (MVP) of an offer network's core functionality. In this prototype, \textbf{users} upload (offer, request) pairs (hereboy called \textbf{ORpairs}) of \textbf{tasks} (either services or goods). An optimization algorithm is run to find and suggent matches, exchanges with parity; the frequency depends on whether a batch or dynamic matching is being used. The first questions are how many users get satisfactory matches and how long this takes, where time is measured in terms of ORpairs added to the offer network. An upper-bound on performance is the maximum possible matching (with $p=1$), and the maximum 2-way exchange matching is a lower-bound.

\end{document}
