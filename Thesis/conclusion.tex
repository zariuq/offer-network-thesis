\documentclass[main.tex]{subfiles}

\begin{document}

\section{Conclusion}
There are many domain-specific barter exchange sites, and a few general ones. The experiments in this thesis suggest that a fast heuristic matching algorithm (GSC with HOR) can overcome the difficulty in bartering: arranging k-user exchanges ($k > 2$), even when the probability users accept matches is not high. The wait time calculated in the thesis depends on how many users ask for exchanges, so a real life offer network that does not reach critical mass will be unbearably slow (however, this is a problem for any social network start-up). Unlike algorithms that find maximal solutions on the whole network, a GSC variant can easily be extended to a decentralized offer network, which is also promising.

Currency allows anything to be exchange for anything, provided there is price parity. This parity, price, is typically understood to be set globally in the marketplace by negotiating to buy for the lowest price possible or sell for the highest price possible, and these prices are influenced by the supply and demand in the market. In the offer network's money-less paradigm, there must be direct evidence for the parity of tasks in an exchange, regardless of the dynamics of the rest of the offer network. This feature is important. First, because the exchange value of a task/good is more concrete: the set of tasks in cycles with $task_a$ are, ostensibly, of equal value. Second, users may have (subjective) mutually exclusive classes of tasks they are willing to exchange. For example, an author may exchange a book review for another favor but not for money (however will work for money in other ways). Another example is, of course, kidney exchange: they are not allowed to be exchanged for money, or nearly anything other than kidneys (or other organs). Thus the direct evidence of task parity in offer networks allows different classes of tasks to be exchanged in one market.

The counter-intuitive performance gains with moderate acceptance probability point toward directions for improvement, all of which increase the ways in which ORpairs can be matched. Increasing the ways in which ORpairs can be matched makes an offer network's performance more like that of a currency-based market. Abbassi \cite{Abb2} try to do this by modeling asynchronous exchange with a credit system; however this becomes another form of money and loses the above feature. When HOR is used, an ORpair can facilitate one match and then become part of a held ORpair to facilitate another match. This allows more flexibility in exchanges while maintaining evidence of parity. The method of using waiting priority queues for pure \textbf{requests} and gift chains to model asynchronous exchange decribed in Section \ref{sec:horquee}, a generalization of HOR, maintains parity and allows more types of exchanges as gifts are, in essence, treated like one unit of task-typed currency. Similarity based task grouping and ORpairs with ORs of offers and requests also make more matches possible. The potentials method of Dickerson \cite{Dick} tries to predict which parts of the offer network will be more useful later and save them: while greedy matching is good, matching everything possible greedily produces poorer results. Even doing this randomly, via low acceptance probability, seems according to this thesis' experiments benefitial.

The experiments in this thesis indicate that GSC with HOR makes offer networks feasible, with respect to total matched ORpairs and the average wait time. Further experimentation is advised. Moreover, there are indications significant improvements can be made by improving the model used for matching in the offer network, much as Dickerson \cite{Dick} \cite{Dick3} and Jia \cite{Jia1} found that approximation algorithms that take acceptance probability into account can out-perform optimal algorithms that don't, and as MAX-WEIGHT performs poorly without HOR.

\end{document}
