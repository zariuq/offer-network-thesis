\documentclass[main.tex]{subfiles}

\begin{document}

\section{Matching Algorithms}
The matching problem is solved on the static graph at a certain point in time. One goal for the optimization algorithms is to find the maximum edge-disjoint cycle covering, which can be solved as Min-D-DCC is solved \cite{Man1} \cite{Bir}. The section \ref{bima} discusses this polynomial time algorithm.

In the case that the optimal soultion contains long cycles, the matching problem is to find the maximum edge-disjoint covering with cycles of $\leq$ edges. This problem is NP-Hard and APX-Complete \cite{Bir}. This is desirable because

Both graph algorithms and integer linear programming (ILP) algorithms can be used. Some additional features may be easier to add in the ILP case , which is a common, reliable choice in the field despite being NP-complete; however additional constraints such as desiring small cycles make the problem NP-Hard.

\subsection{Matching via Perfect Bipartite Matching}\label{bima}

An algorithm to find perfect weighted matchings in bipartite graph can be used to discover the optimal match in polynomial time \cite{Bir}. The bipartite graph has one side for Offers and one for Requests, thus each (offer=$t_a$, request=$t_b$) pair has two nodes created with edge weight $0$. Next an edge is created between the offfer node and every request node for $t_a$ with weight $> 0$. Then a maximum weight perfect matching algorithm can be used.

The positive weight can be used to assign preferences to some (offer, request) pairs over others, perhaps due to user preferences (soft constraints).

The drawback is that long-cycles are possible, and in the dense graph for kidney exchange, very likely. Each user reserves the right to refuse a suggested match, so longer matches are more likely to be rejected, and in the kidney case, require organizing dozens of simultaneous surgeries. However, bounding the match-size results in an NP-hard problem \cite{Bir}.

\subsection{Matching via Weighted Boolean Optimization}

One linear programming formulation of the problem is for weighted boolean optimization \cite{Mar1}.

The variables:
\begin{itemize}
  \item Let $x_{abt}$ denote $u_a$ doing $t_t$ for $u_b$
  \item Let $r_{at}$ denote a task $u_a$ requests for offered task $t_t$.

        That is, an edge: $(t_t, r_{at})$ : $u_a$
  \item Let $s_{at}$ be a selection varable indicating whether $u_a$'s request $t_t$ is satisfied
  \item Let $w_{abt}$ be a weight describing $u_a$'s satisfaction with $u_b$ fulfilling request $t_t$
\end{itemize}

Denote $U$ the set of users and $T$ the set of tasks
The constraints for each user $a \in U$ and $t \in T$:
\begin{enumerate}
  \item $\sum_{b \in U} x_{abt} \leq 1$
  \item $sum_{b \in U} x_{abt} = \sum{b \in U} x_{bar_{at}}$
  \item $s_{at} + \sum_{b \in U} x_{bar_{at}} > 1$
\end{enumerate}

Minimize:
  $$\sum_{a \in U, b \in U, t \in T} w_{abt} s_{at}$$

One the positive side, a lot of work has gone into good linear program and SAT solvers despite the problem being NP-complete. Furthermore, additional features are easy to add into the formulation.

A big potential drawback is that there are $|U \times T|$ constraints (all sums over $U$) to be dealt with. Dealing with kidney exchange, \cite{Abr1} found the machines ran out of memory and needed to use a more complex, incremental column generation approach. However, they did formulate the problem differently for LP.

\end{document}
