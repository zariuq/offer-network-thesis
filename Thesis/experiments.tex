\documentclass[main.tex]{subfiles}

\begin{document}

\subsection{Performance Metrics}
The two standard metrics are:
\begin{itemize}
  \item Number of ORpairs satisfactorily matched.
  \item Wait time of ORpairs that are matched.
\end{itemize}

Next in when using the HOR approach, the amount of times a hanging ORpair is successively created is a measure of performance and feasibility.

\section{Experiments}



\section{Preliminary exploration}
Prior to running grid searches over parametre space, I am informally testing different settings.

First, when the scale free graph is generated based on in and out degree independently, there are few matches.

Nodes do not seem to be held more than 1 time much, and more than 2 fairly rarely (say when adding 1000 ORpairs at 15 node steps).

When $p=1$, the matching algorithm performs better when run less frequently (i.e. larger step sizes); whereas when the step size is shorter and $p > 0.8$, the hanging ORpairs approach matches almost as many satisfactorily.

The greedy shortest cycle algorithm performs almost as well as the optimal algorithm. Moreover, the optimal algorithm without hanging ORpairs performs quite poorly with $p < 0.9$. On the other hand, the greedy shortest cycle algorithm's performance is not hurt much with short enough steps.

\end{document}
